% Options for packages loaded elsewhere
\PassOptionsToPackage{unicode}{hyperref}
\PassOptionsToPackage{hyphens}{url}
%
\documentclass[
]{article}
\usepackage{amsmath,amssymb}
\usepackage{iftex}
\ifPDFTeX
  \usepackage[T1]{fontenc}
  \usepackage[utf8]{inputenc}
  \usepackage{textcomp} % provide euro and other symbols
\else % if luatex or xetex
  \usepackage{unicode-math} % this also loads fontspec
  \defaultfontfeatures{Scale=MatchLowercase}
  \defaultfontfeatures[\rmfamily]{Ligatures=TeX,Scale=1}
\fi
\usepackage{lmodern}
\ifPDFTeX\else
  % xetex/luatex font selection
\fi
% Use upquote if available, for straight quotes in verbatim environments
\IfFileExists{upquote.sty}{\usepackage{upquote}}{}
\IfFileExists{microtype.sty}{% use microtype if available
  \usepackage[]{microtype}
  \UseMicrotypeSet[protrusion]{basicmath} % disable protrusion for tt fonts
}{}
\makeatletter
\@ifundefined{KOMAClassName}{% if non-KOMA class
  \IfFileExists{parskip.sty}{%
    \usepackage{parskip}
  }{% else
    \setlength{\parindent}{0pt}
    \setlength{\parskip}{6pt plus 2pt minus 1pt}}
}{% if KOMA class
  \KOMAoptions{parskip=half}}
\makeatother
\usepackage{xcolor}
\usepackage[margin=1in]{geometry}
\usepackage{color}
\usepackage{fancyvrb}
\newcommand{\VerbBar}{|}
\newcommand{\VERB}{\Verb[commandchars=\\\{\}]}
\DefineVerbatimEnvironment{Highlighting}{Verbatim}{commandchars=\\\{\}}
% Add ',fontsize=\small' for more characters per line
\usepackage{framed}
\definecolor{shadecolor}{RGB}{248,248,248}
\newenvironment{Shaded}{\begin{snugshade}}{\end{snugshade}}
\newcommand{\AlertTok}[1]{\textcolor[rgb]{0.94,0.16,0.16}{#1}}
\newcommand{\AnnotationTok}[1]{\textcolor[rgb]{0.56,0.35,0.01}{\textbf{\textit{#1}}}}
\newcommand{\AttributeTok}[1]{\textcolor[rgb]{0.13,0.29,0.53}{#1}}
\newcommand{\BaseNTok}[1]{\textcolor[rgb]{0.00,0.00,0.81}{#1}}
\newcommand{\BuiltInTok}[1]{#1}
\newcommand{\CharTok}[1]{\textcolor[rgb]{0.31,0.60,0.02}{#1}}
\newcommand{\CommentTok}[1]{\textcolor[rgb]{0.56,0.35,0.01}{\textit{#1}}}
\newcommand{\CommentVarTok}[1]{\textcolor[rgb]{0.56,0.35,0.01}{\textbf{\textit{#1}}}}
\newcommand{\ConstantTok}[1]{\textcolor[rgb]{0.56,0.35,0.01}{#1}}
\newcommand{\ControlFlowTok}[1]{\textcolor[rgb]{0.13,0.29,0.53}{\textbf{#1}}}
\newcommand{\DataTypeTok}[1]{\textcolor[rgb]{0.13,0.29,0.53}{#1}}
\newcommand{\DecValTok}[1]{\textcolor[rgb]{0.00,0.00,0.81}{#1}}
\newcommand{\DocumentationTok}[1]{\textcolor[rgb]{0.56,0.35,0.01}{\textbf{\textit{#1}}}}
\newcommand{\ErrorTok}[1]{\textcolor[rgb]{0.64,0.00,0.00}{\textbf{#1}}}
\newcommand{\ExtensionTok}[1]{#1}
\newcommand{\FloatTok}[1]{\textcolor[rgb]{0.00,0.00,0.81}{#1}}
\newcommand{\FunctionTok}[1]{\textcolor[rgb]{0.13,0.29,0.53}{\textbf{#1}}}
\newcommand{\ImportTok}[1]{#1}
\newcommand{\InformationTok}[1]{\textcolor[rgb]{0.56,0.35,0.01}{\textbf{\textit{#1}}}}
\newcommand{\KeywordTok}[1]{\textcolor[rgb]{0.13,0.29,0.53}{\textbf{#1}}}
\newcommand{\NormalTok}[1]{#1}
\newcommand{\OperatorTok}[1]{\textcolor[rgb]{0.81,0.36,0.00}{\textbf{#1}}}
\newcommand{\OtherTok}[1]{\textcolor[rgb]{0.56,0.35,0.01}{#1}}
\newcommand{\PreprocessorTok}[1]{\textcolor[rgb]{0.56,0.35,0.01}{\textit{#1}}}
\newcommand{\RegionMarkerTok}[1]{#1}
\newcommand{\SpecialCharTok}[1]{\textcolor[rgb]{0.81,0.36,0.00}{\textbf{#1}}}
\newcommand{\SpecialStringTok}[1]{\textcolor[rgb]{0.31,0.60,0.02}{#1}}
\newcommand{\StringTok}[1]{\textcolor[rgb]{0.31,0.60,0.02}{#1}}
\newcommand{\VariableTok}[1]{\textcolor[rgb]{0.00,0.00,0.00}{#1}}
\newcommand{\VerbatimStringTok}[1]{\textcolor[rgb]{0.31,0.60,0.02}{#1}}
\newcommand{\WarningTok}[1]{\textcolor[rgb]{0.56,0.35,0.01}{\textbf{\textit{#1}}}}
\usepackage{longtable,booktabs,array}
\usepackage{calc} % for calculating minipage widths
% Correct order of tables after \paragraph or \subparagraph
\usepackage{etoolbox}
\makeatletter
\patchcmd\longtable{\par}{\if@noskipsec\mbox{}\fi\par}{}{}
\makeatother
% Allow footnotes in longtable head/foot
\IfFileExists{footnotehyper.sty}{\usepackage{footnotehyper}}{\usepackage{footnote}}
\makesavenoteenv{longtable}
\usepackage{graphicx}
\makeatletter
\def\maxwidth{\ifdim\Gin@nat@width>\linewidth\linewidth\else\Gin@nat@width\fi}
\def\maxheight{\ifdim\Gin@nat@height>\textheight\textheight\else\Gin@nat@height\fi}
\makeatother
% Scale images if necessary, so that they will not overflow the page
% margins by default, and it is still possible to overwrite the defaults
% using explicit options in \includegraphics[width, height, ...]{}
\setkeys{Gin}{width=\maxwidth,height=\maxheight,keepaspectratio}
% Set default figure placement to htbp
\makeatletter
\def\fps@figure{htbp}
\makeatother
\setlength{\emergencystretch}{3em} % prevent overfull lines
\providecommand{\tightlist}{%
  \setlength{\itemsep}{0pt}\setlength{\parskip}{0pt}}
\setcounter{secnumdepth}{-\maxdimen} % remove section numbering
\ifLuaTeX
  \usepackage{selnolig}  % disable illegal ligatures
\fi
\usepackage{bookmark}
\IfFileExists{xurl.sty}{\usepackage{xurl}}{} % add URL line breaks if available
\urlstyle{same}
\hypersetup{
  pdftitle={myscript},
  pdfauthor={Taylor PD},
  hidelinks,
  pdfcreator={LaTeX via pandoc}}

\title{myscript}
\author{Taylor PD}
\date{2025-01-07}

\begin{document}
\maketitle

\href{https://github.com/Taylorpd4/BIOL432_Assignment1TP}{GitHub
Repository}

\subsection{Introduction}\label{introduction}

This report shows the introductory process including the necessary
skills that are applyed when analyzing biological measurement data. It
executes two R scripts, \texttt{dataGenerato.R} and
\texttt{volumeEstimato.R}, to create and enhance the dataset. This
report also includes data sorting, summary statistics, and
visualizations.

\begin{Shaded}
\begin{Highlighting}[]
\CommentTok{\# Load required libraries}
\FunctionTok{library}\NormalTok{(dplyr)}
\end{Highlighting}
\end{Shaded}

\begin{verbatim}
## 
## Attaching package: 'dplyr'
\end{verbatim}

\begin{verbatim}
## The following objects are masked from 'package:stats':
## 
##     filter, lag
\end{verbatim}

\begin{verbatim}
## The following objects are masked from 'package:base':
## 
##     intersect, setdiff, setequal, union
\end{verbatim}

\begin{Shaded}
\begin{Highlighting}[]
\FunctionTok{library}\NormalTok{(ggplot2)}
\FunctionTok{library}\NormalTok{(tidyr)}
\end{Highlighting}
\end{Shaded}

\subsection{Step 1: Execute R Scripts}\label{step-1-execute-r-scripts}

\subsubsection{1.1 Generate Data}\label{generate-data}

The script \texttt{dataGenerato.R} creates a dataset with 100
observations, including species names, limb measurements, and observer
names. the dataset is saved as \texttt{measurements.csv}.

\begin{Shaded}
\begin{Highlighting}[]
\FunctionTok{source}\NormalTok{(}\StringTok{"dataGenerato.R"}\NormalTok{)}
\end{Highlighting}
\end{Shaded}

\begin{verbatim}
## Data has been generated and saved to measurements.csv
\end{verbatim}

\subsubsection{1.2 Estimate Volume}\label{estimate-volume}

The script \texttt{volumeEstimato.R} calculates the limb volume for each
observation. It adds this data as a new column to
\texttt{measurements.csv}.

\begin{Shaded}
\begin{Highlighting}[]
\FunctionTok{source}\NormalTok{(}\StringTok{"volumeEstimato.R"}\NormalTok{)}
\end{Highlighting}
\end{Shaded}

\begin{verbatim}
## Volume column has been added to measurements.csv
\end{verbatim}

\subsection{Step 2: Load and Prepare
Data}\label{step-2-load-and-prepare-data}

\begin{Shaded}
\begin{Highlighting}[]
\NormalTok{data }\OtherTok{\textless{}{-}} \FunctionTok{read.csv}\NormalTok{(}\StringTok{"measurements.csv"}\NormalTok{)}
\end{Highlighting}
\end{Shaded}

\subsubsection{2.1 Sort Data by Species, Observer, and Limb
Volume}\label{sort-data-by-species-observer-and-limb-volume}

\begin{Shaded}
\begin{Highlighting}[]
\NormalTok{sorted\_data }\OtherTok{\textless{}{-}}\NormalTok{ data }\SpecialCharTok{\%\textgreater{}\%} \FunctionTok{arrange}\NormalTok{(Species, Observer, Volume)}
\FunctionTok{head}\NormalTok{(sorted\_data)}
\end{Highlighting}
\end{Shaded}

\begin{verbatim}
##            Species Limb_Width Limb_Length Observer   Volume
## 1 Corvus splendens   2.651710    28.29353    GayaS 156.2532
## 2 Corvus splendens   3.432325    43.54084    GayaS 402.8686
## 3 Corvus splendens   4.536443    27.32467    GayaS 441.6476
## 4 Corvus splendens   5.535917    21.08971    GayaS 507.6212
## 5 Corvus splendens   5.853895    25.05161    GayaS 674.2414
## 6 Corvus splendens   6.184931    29.19606    GayaS 877.1700
\end{verbatim}

\subsubsection{2.2 Create Table of Average Volume by
Species}\label{create-table-of-average-volume-by-species}

\begin{Shaded}
\begin{Highlighting}[]
\NormalTok{average\_volume }\OtherTok{\textless{}{-}}\NormalTok{ data }\SpecialCharTok{\%\textgreater{}\%} 
  \FunctionTok{group\_by}\NormalTok{(Species) }\SpecialCharTok{\%\textgreater{}\%} 
  \FunctionTok{summarise}\NormalTok{(}\AttributeTok{Average\_Volume =} \FunctionTok{mean}\NormalTok{(Volume, }\AttributeTok{na.rm =} \ConstantTok{TRUE}\NormalTok{))}

\NormalTok{knitr}\SpecialCharTok{::}\FunctionTok{kable}\NormalTok{(average\_volume, }\AttributeTok{caption =} \StringTok{"Average Limb Volume by Species"}\NormalTok{)}
\end{Highlighting}
\end{Shaded}

\begin{longtable}[]{@{}lr@{}}
\caption{Average Limb Volume by Species}\tabularnewline
\toprule\noalign{}
Species & Average\_Volume \\
\midrule\noalign{}
\endfirsthead
\toprule\noalign{}
Species & Average\_Volume \\
\midrule\noalign{}
\endhead
\bottomrule\noalign{}
\endlastfoot
Corvus splendens & 589.1492 \\
Equus caballus & 623.4305 \\
Formicidae & 632.5103 \\
Hippopotamus amphibius & 684.0618 \\
Urusus americanus & 559.7205 \\
\end{longtable}

\subsubsection{2.3 Create Table of Observation Counts by Species and
Observer}\label{create-table-of-observation-counts-by-species-and-observer}

\begin{Shaded}
\begin{Highlighting}[]
\NormalTok{observation\_counts }\OtherTok{\textless{}{-}}\NormalTok{ data }\SpecialCharTok{\%\textgreater{}\%} 
  \FunctionTok{group\_by}\NormalTok{(Species, Observer) }\SpecialCharTok{\%\textgreater{}\%} 
  \FunctionTok{summarise}\NormalTok{(}\AttributeTok{Count =} \FunctionTok{n}\NormalTok{(), }\AttributeTok{.groups =} \StringTok{"drop"}\NormalTok{)}

\NormalTok{knitr}\SpecialCharTok{::}\FunctionTok{kable}\NormalTok{(observation\_counts, }\AttributeTok{caption =} \StringTok{"Observation Counts by Species and Observer"}\NormalTok{)}
\end{Highlighting}
\end{Shaded}

\begin{longtable}[]{@{}llr@{}}
\caption{Observation Counts by Species and Observer}\tabularnewline
\toprule\noalign{}
Species & Observer & Count \\
\midrule\noalign{}
\endfirsthead
\toprule\noalign{}
Species & Observer & Count \\
\midrule\noalign{}
\endhead
\bottomrule\noalign{}
\endlastfoot
Corvus splendens & GayaS & 7 \\
Corvus splendens & MaryT & 9 \\
Corvus splendens & NithilaS & 3 \\
Equus caballus & GayaS & 2 \\
Equus caballus & MaryT & 10 \\
Equus caballus & NithilaS & 5 \\
Formicidae & GayaS & 8 \\
Formicidae & MaryT & 6 \\
Formicidae & NithilaS & 6 \\
Hippopotamus amphibius & GayaS & 9 \\
Hippopotamus amphibius & MaryT & 9 \\
Hippopotamus amphibius & NithilaS & 5 \\
Urusus americanus & GayaS & 9 \\
Urusus americanus & MaryT & 8 \\
Urusus americanus & NithilaS & 4 \\
\end{longtable}

\subsection{Step 3: Visualizations}\label{step-3-visualizations}

\subsubsection{3.1 Box Plot of Volume by
Species}\label{box-plot-of-volume-by-species}

\begin{Shaded}
\begin{Highlighting}[]
\FunctionTok{library}\NormalTok{(ggplot2)}
\FunctionTok{ggplot}\NormalTok{(data, }\FunctionTok{aes}\NormalTok{(}\AttributeTok{x =}\NormalTok{ Species, }\AttributeTok{y =}\NormalTok{ Volume, }\AttributeTok{fill =}\NormalTok{ Species)) }\SpecialCharTok{+} 
  \FunctionTok{geom\_boxplot}\NormalTok{() }\SpecialCharTok{+} 
  \FunctionTok{labs}\NormalTok{(}\AttributeTok{title =} \StringTok{"Distribution of Limb Volumes by Species"}\NormalTok{, }\AttributeTok{x =} \StringTok{"Species"}\NormalTok{, }\AttributeTok{y =} \StringTok{"Limb Volume"}\NormalTok{) }\SpecialCharTok{+} 
  \FunctionTok{theme\_minimal}\NormalTok{()}
\end{Highlighting}
\end{Shaded}

\includegraphics{FinalReport_files/figure-latex/unnamed-chunk-8-1.pdf}

\subsubsection{3.2 Multi-Panel Histograms of Volume by
Species}\label{multi-panel-histograms-of-volume-by-species}

\begin{Shaded}
\begin{Highlighting}[]
\FunctionTok{ggplot}\NormalTok{(data, }\FunctionTok{aes}\NormalTok{(}\AttributeTok{x =}\NormalTok{ Volume)) }\SpecialCharTok{+} 
  \FunctionTok{geom\_histogram}\NormalTok{(}\AttributeTok{binwidth =} \DecValTok{5}\NormalTok{, }\AttributeTok{fill =} \StringTok{"blue"}\NormalTok{, }\AttributeTok{color =} \StringTok{"black"}\NormalTok{, }\AttributeTok{alpha =} \FloatTok{0.7}\NormalTok{) }\SpecialCharTok{+} 
  \FunctionTok{facet\_wrap}\NormalTok{(}\SpecialCharTok{\textasciitilde{}}\NormalTok{ Species, }\AttributeTok{scales =} \StringTok{"free"}\NormalTok{) }\SpecialCharTok{+} 
  \FunctionTok{labs}\NormalTok{(}\AttributeTok{title =} \StringTok{"Frequency Distribution of Limb Volume by Species"}\NormalTok{, }\AttributeTok{x =} \StringTok{"Limb Volume"}\NormalTok{, }\AttributeTok{y =} \StringTok{"Frequency"}\NormalTok{) }\SpecialCharTok{+} 
  \FunctionTok{theme\_minimal}\NormalTok{()}
\end{Highlighting}
\end{Shaded}

\includegraphics{FinalReport_files/figure-latex/unnamed-chunk-9-1.pdf}

\subsection{Conclusion}\label{conclusion}

This report provides a complete workflow for analyzing biological
measurement data.

```

\end{document}
